\documentclass[letterpaper,12pt,leqno]{article}
\usepackage{paper,math,notes}

\begin{document}

\title{Mathematical Methods for Macroeconomics: Exercises}
\author{Pascal Michaillat}
\date{}

\begin{titlepage}
\maketitle
\end{titlepage}

\section*{Dynamic Programming}

\subsection*{Exercise 1.}

Consider the following optimal growth problem: Given initial capital $k_{0}>0$, choose consumption $\bc{c_{t}} _{t =0}^{+\infty}$ to maximize utility
\begin{equation*}
\sum_{t=0}^{\infty}\b ^{t}\cdot \ln{c_{t}}
\end{equation*}
subject to the resource constraint
\begin{equation*}
k_{t+1}=A\cdot k_{t}^{\a}-c_{t}.
\end{equation*}
The parameters satisfy $0<\b<1,\;A>0,\;0<\a <1.$

\begin{enumerate}
\item Derive the optimal law of motion of consumption $c_{t}$ using a Lagrangian.
\item Identify the state variable and the control variable. 
\item Write down the Bellman equation.
\item Derive the following Euler equation: 
\begin{equation*}
c_{t+1}=\b\cdot  \a\cdot  A\cdot k_{t+1}^{\a -1}\cdot c_{t}.
\end{equation*}

\item Derive the first two value functions, $V_{1}(k)$ and  $V_{2}(k)$, obtained by iteration on the Bellman equation starting with the value function $V_{0}\bp{k} \equiv 0$. 
\item The process of determining the value function by iterations using the Bellman equation is commonly used to solve dynamic programs numerically. The algorithm is called \textit{value function iteration}. For this optimal growth problem, one can show show using value function iteration that the value function is
\[V\bp{k} =\kappa +\frac{\ln{k^{\a}}}{1-\a\cdot \b},\]
where $\k$ is a constant. Using the Bellman equation, determine the policy function $k'(k)$ associated with this value function.
\item In light of these results, for which reasons would you prefer to use the dynamic-programming approach instead of the Lagrangian approach to solve the optimal growth problem? And for which reasons would you prefer to use the Lagrangian approach instead of the dynamic-programming approach?
\end{enumerate}

\subsection*{Exercise 2.}

Consider the problem of choosing consumption $\bc{c_{t}}_{t=0}^{+\infty}$ to maximize expected utility
\begin{equation*}
\E_{0}\sum_{t=0}^{+\infty}\b^{t}\cdot u\bp{c_{t}}
\end{equation*}
subject to the budget constraint
\begin{equation*}
c_{t}+p_{t}\cdot s_{t+1}=\bp{d_{t}+p_{t}}\cdot s_{t}.
\end{equation*}
$d_{t}$ is the dividend paid out for one share of the asset, $p_{t}$ is the price of one share of the asset, and $s_{t}$ is the number of shares of the asset held at the beginning of period $t$. In equilibrium, the price $p_{t}$ of one share is solely a function of dividends $d_{t}$. Dividends can only take two values $d_{l}$ and $d_{h}$, with $0<d_{l}<d_{h}$. Dividends follow a Markov process with transition probabilities 
\begin{equation*}
\P\bp{d_{t+1}=d_{l}\mid d_{t}=d_{l}} =\P \bp{d_{t+1}=d_{h}\mid d_{t}=d_{h}} =\rho
\end{equation*}
with $1>\rho >0.5.$

\begin{enumerate}
\item Identify state and control variables. 
\item Write down the Bellman equation.
\item Derive the following Euler equation: 
\begin{equation*}
p_{t}\cdot u'\bp{c_{t}} =\b\cdot  \E{\bp{d_{t+1}+p_{t+1}} \cdot u'\bp{c_{t+1}} \mid d_{t}} .
\end{equation*}

\item Suppose that $u\bp{c} =c$. Show that the asset price is higher when the current dividend is high.
\end{enumerate}

\subsection*{Exercise 3.}

Consider the following optimal growth problem: Given initial capital $k_{0}>0$, choose consumption and labor $\bc{c_{t},l_{t}}_{t=0}^{+\infty}$ to maximize utility
\begin{equation*}
\sum_{t=0}^{+\infty}\b^{t}\cdot u\bp{c_{t},l_{t}}
\end{equation*}
subject to the law of motion of capital
\begin{align*}
k_{t+1}&=A_{t}\cdot f\bp{k_{t},l_{t}} -c_{t}.
\end{align*}
In addition, we impose $0\leq l_{t}\leq 1$. The discount factor $\b \in \bp{0,1} $. The function $f$ is increasing and concave in both arguments. The function $u$ is increasing and concave in $c$, decreasing and convex in $l$. 

\paragraph{Deterministic case} First, suppose $A_{t}=1$ for all $t$.

\begin{enumerate}
\item What are the state and control variables?
\item  Write down the Bellman equation.
\item Derive the following optimality conditions: 
\begin{align*}
\pd{u\bp{c_{t},l_{t}}}{ c_{t}} &=\b \cdot \pd{u\bp{c_{t+1},l_{t+1}}}{c_{t+1}} \cdot \pd{f\bp{k_{t+1},l_{t+1}}}{ k_{t+1}}\\
\pd{u\bp{c_{t},l_{t}}}{ c_{t}}\cdot \pd{f\bp{k_{t},l_{t}}}{l_{t}} &=-\pd{u\bp{c_{t},l_{t}}}{l_{t}}.
\end{align*}
\item Suppose that the production function $f\bp{k,l} =k^{\a}\cdot l^{1-\a}$. Determine the ratios $c/k$ and $l/k$ in steady state.
\end{enumerate}

\paragraph{Stochastic case} Now, suppose $A_{t}$ is a stochastic process that takes values $A_{1}$ and $A_{2}$ with the following probability: 
\begin{equation*}
\P{A_{t+1}=A_{1}\mid A_{t}=A_{1}} =\P{A_{t+1}=A_{2}\mid A_{t}=A_{2}} =\rho .
\end{equation*}

\begin{enumerate}\setcounter{enumi}{4}
\item Write down the Bellman equation.
\item Derive the optimality conditions.
\end{enumerate}

\section*{Optimal Control}

\subsection*{Exercise 4.}

Consider the following optimal growth problem: Given initial capital $k_{0}>0$, choose a consumption path $\bc{c_{t}}_{t\geq 0}$ to maximize utility
\begin{align*}
\int_{0}^{\infty}e^{-\rho\cdot  t} \cdot \ln{c_{t}} dt 
\end{align*}
subject to the law of motion of capital
\begin{align*}
\dot{k}_{t} &=f\bp{k_{t}} -c_{t}-\d \cdot k_{t}.
\end{align*}
The discount factor $\rho>0$, and the production function $f$ satisfies
 \[f\bp{k} =A\cdot k^{\a},\]
 where  $\a \in \bp{0,1}$ and $A>0$.

\begin{enumerate}
\item Write down the present-value Hamiltonian.
\item Show that the Euler equation is
\begin{align*}
\frac{\dot{c}_{t}}{c_{t}} &=\a \cdot  A \cdot k_{t}^{\a -1}-\bp{\d +\rho}.
\end{align*}

\item Solve for the steady state of the system.
\end{enumerate}

\subsection*{Exercise 5.}

Consider the following investment problem: Given initial capital $k_{0}$, choose the investment path $\bc{i_{t}} _{t \geq 0}$ to maximize profits
\begin{align*}
\int_{0}^{\infty} e^{-r\cdot t}\bs{f\bp{k_{t}} -i_{t}-\frac{\chi}{2}\cdot \bp{\frac{i_{t}^{2}}{k_{t}}}} dt 
\end{align*}
subject to the law of motion of capital (we assume no capital depreciation)
\[\dot{k}_{t} =i_{t}.\]
The interest rate $r>0$, the capital adjustment cost $\chi>0$, and the production function $f$ satisfies $f'>0$ and $f''<0$.

\begin{enumerate}
\item Write down the current-value Hamiltonian. 
\item Use the optimality conditions for the current-value Hamiltonian to derive the following differential equations:
\begin{align*}
\dot{k}_{t} &=\bp{\frac{q_{t}-1}{\chi}}\cdot  k_{t} \\
\dot{q}_{t} &=r\cdot q_{t}-f'\bp{k_{t}} -\frac{1}{2\cdot \chi}\bp{q_{t}-1}^{2}
\end{align*}

\item Solve for the steady state.
\end{enumerate}

\section*{Differential Equations}

\subsection*{Exercise 6.}

Find the solution of the initial value problem 
\begin{align*}
\dot{a}(t) &=r\cdot a(t) +s \\
a\bp{0} &=a_{0}
\end{align*}
where both $r$ and $s$ are known constant.

\subsection*{Exercise 7.}

Find the solution of the initial value problem 
\begin{align*}
\dot{a}(t) &=r(t)\cdot a(t) +s(t) \\
a\bp{0} &=a_{0}
\end{align*}
where both  $r(t)$ and $s(t)$ are known functions of $t.$

\subsection*{Exercise 8.}

Consider the linear system of FODEs given by
\begin{equation*}
\bm{\dot{x}}(t)=\bs{
\begin{array}{ll}
1 & 1 \\ 
4 & 1
\end{array}} \bm{x}(t).
\end{equation*}
\begin{enumerate}
\item Find the general solution of the system.
\item What would you need to find a specific solution of the system?
\item Draw the trajectories of the system.
\end{enumerate}

\subsection*{Exercise 9.}

Consider the initial value problem 
\begin{align*}
\dot{k}(t) &=s\cdot f\bp{k(t)} -\d\cdot  k(t) \\
k\bp{0} &=k_{0}
\end{align*}
where the saving rate $s\in \bp{0,1} $, the capital depreciation rate $\d \in \bp{0,1}$, and the production function $f$ satisfies the \textit{Inada conditions}. That is, $f$ is continuously differentiable and 
\begin{align*}
f(0)&=0\\
f'(x)&>0\\
f''(x)&<0\\
\lim_{x\to 0} f'(x)&=+\infty\\
\lim_{x\to +\infty} f'(x)&=0.
\end{align*}

\begin{enumerate}
\item Give a production function $f$ that satisfies the Inada conditions.
\item Find the steady state of the system.
\item Draw the dynamic path of $k(t) $ and show that it converges to the steady state.
\end{enumerate}

\subsection*{Exercise 10.}

The solution of the problem studied in Exercise 4 is characterized by a system of two nonlinear first-order differential equations:
\begin{align*}
\dot{k}_{t} &=f\bp{k_{t}} -c_{t}-\d \cdot k_{t}\\
\frac{\dot{c}_{t}}{c_{t}} &=\a \cdot  A \cdot k_{t}^{\a -1}-\bp{\d +\rho}.
\end{align*}
The first FODE is the law of motion of capital. The second FODE is the Euler equation, which describes the optimal path of consumption over time.


\begin{enumerate}
\item Draw the phase diagram of the system.
\item Linearize the system around its steady state.
\item Show that the steady state is a saddle point  locally.
\item Suppose the economy is in steady state at time $t_{0}$ and there is an unanticipated decrease in the discount factor $\rho$. Show on your phase diagram the transition dynamics of the model.
\end{enumerate}

\subsection*{Exercise 11.}

The solution of the investment problem studied in Exercise 5 is characterized by a system of two nonlinear first-order differential equations:
\begin{align*}
\dot{k}_{t} &=\bp{\frac{q_{t}-1}{\chi}}\cdot  k_{t} \\
\dot{q}_{t} &=r\cdot q_{t}-f'\bp{k_{t}} -\frac{1}{2\cdot \chi}\bp{q_{t}-1}^{2}.
\end{align*}
The first FODE is the law of motion of capital $k_{t}$. The second FODE is the law of motion of the co-state variable $q_{t}$.

\begin{enumerate}
\item Draw the phase diagram.
\item Show that the steady state is a saddle point locally.
\end{enumerate}

\subsection*{Exercise 12.}

Consider a discrete time version of the typical growth model:
\begin{align*}
k(t+1) &=f\bp{k(t)} -c(t) +\bp{1-\d}\cdot  k(t) \\
c(t+1) &=\b\cdot  \bs{ 1+f'\bp{k(t)} -\d }\cdot  c(t) .
\end{align*} 
The discount factor $\b \in \bp{0,1}$, the rate of depreciation of capital $\d \in \bp{0,1}$, initial capital $k_{0}$ is given, and the production function $f$ satisfies the Inada conditions. These two equations are a system of first-order difference equations. Whereas a system of first-order differential equations relates $\bm{\dot{x}}(t) $ to $\bm{x}(t)$, a system of first-order difference equations relate $\bm{x}(t+1) $ to $\bm{x}(t)$.

In this exercise, we will see that we can study a system of first-order difference equations with the tools that we used to study  systems of first-order differential equations. In particular, we can use phase diagrams to understand the dynamics of the system.

\begin{enumerate}
\item Construct a phase diagram for the system. First, define 
\begin{align*}
\D k & \equiv k(t+1) -k(t) , \\
\D c & \equiv c(t+1) -c(t) .
\end{align*}
Second, draw the $\D k=0$ locus and the $\D c=0$ locus on the $(k,c)$ plane. Finally, find the steady state as the intersection of the $\D k=0$ locus and the $ \D c=0$ locus.
\item Show that the steady state is a saddle point in the phase diagram.
\end{enumerate}

\subsection*{Exercise 13.}

We consider the following optimal growth problem. Given initial human capital $h_{0}$ and initial physical capital $k_{0}$, choose consumption $c(t) $ and labor $l(t) $ to maximize utility
\begin{equation*}
\int_{0}^{\infty}e^{-\rho\cdot  t}\cdot \ln{c} dt
\end{equation*}
subject to
\begin{align*}
\dot{k}_{t} &=y_{t}-c_{t}-\d\cdot  k_{t} \\
\dot{h}_{t} &=B\cdot \bp{1-l_{t}}\cdot  h_{t}.
\end{align*}
Output $y_{t}$ is defined by
\[y_{t}\equiv A\cdot k_{t}^{\a}\cdot \bp{l_{t}\cdot h_{t}} ^{\b}.\]
We also impose that $0 \leq l_{t}\leq 1$. The discount factor $\rho>0$, the rate of depreciation of physical capital $\d>0$, the constants $A>0$ and $B>0$, and the production function parameters $\a\in \bp{0,1}$ and $\b \in\bp{0,1}$.

\begin{enumerate}
\item Give state and control variables.
\item Write down the present-value Hamiltonian for this problem.
\item Derive the optimality conditions. 
\item Show that the growth rate of consumption $c(t)$ is
\begin{equation*}
\frac{\dot{c}}{c}=\frac{\a\cdot  y}{k}-\bp{\d +\rho} .
\end{equation*}
\item From now on, we assume that $B=0$. Show that $l=1$.
\item Draw the phase diagram in the $(k,c)$ plane.
\item Show on the diagram that the steady state of the system is a saddle point.
\item Derive the Jacobian of the system.
\item Show that the steady state of the system is a saddle point.
\end{enumerate}


\end{document}