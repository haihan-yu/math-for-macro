\documentclass[letterpaper,12pt,leqno]{article}
\usepackage{paper,math,notes}

\begin{document}

\title{Mathematical Methods for Macroeconomics: One-Hour Exam}
\author{Pascal Michaillat}
\date{}

\begin{titlepage}
\maketitle
\end{titlepage}

\section*{Question 1. (50 pts)}

Let $\a \in (0,1)$, $\d  \in (0,1)$, $\rho \in (0,1)$, and $\s>0$. Impose that $\rho+\d<1$. Given $k(0)$, we want to find the function $c(t) $ to maximize 
\begin{equation*}
\int_{0}^{+\infty }e^{-\rho\cdot  t}\cdot \frac{c(t)^{1-\s}-1}{1-\s} dt,
\end{equation*}
subject to the law of motion
\begin{equation*}
\dot{k}(t) =k(t)^{\a}-c(t)-\d \cdot k(t).
\end{equation*}

\begin{enumerate}

\item (20 pts) Which variable do you choose as a state variable? Which variable do you choose as a control variable? Write down the current-value Hamiltonian and derive the optimality conditions.

\item (5 pts) The Euler equation is the first-order differential equation that characterizes the optimal function $c(t)$. Determine the Euler equation.

\item (10 pts) Suppose $\a =1$ and $\s =1$. Show that the system describing the optimal functions $\{k(t),c(t)\}$ reduces to a linear, homogenous system of first-order differential equations. Show that the system is unstable by computing the eigenvalues. 

\item (15 pts) Suppose $\a <1$ and $\s >0$. Show that the system describing the optimal functions $\{k(t),c(t)\}$ reduces to a nonlinear system of first-order differential equations. Use a phase-diagram to show that the steady state of the system is a saddle point. Explain how you draw the phase diagram.

\end{enumerate}

\section*{Question 2. (50 pts)}

Let $\b \in (0,1)$ and $r>0$. Given $k_{0}>0$, we want to find a collection of sequences $\{c_{t},k_{t+1}\}_{t=0}^{+\infty}$ to maximize 
\begin{equation*}
\sum_{t=0}^{\infty }\b^{t} \cdot \ln(c_{t}), 
\end{equation*}
subject to the constraints
\begin{equation*}
k_{t+1}=(1+r)\cdot k_{t}-c_{t}
\end{equation*}
for all $t\geq 0$.

\paragraph{Lagragian} We first solve the maximization problem using the Lagrangian method.

\begin{enumerate}
\item (5 pts) Write down the Lagrangian of the problem.
\item (5 pts) Derive the first-order condition(s) of the maximization problem.
\item (5 pts) Derive the Euler equation.
\end{enumerate}

\paragraph{Dynamic Programming} Next we solve the maximization problem using the dynamic programming method.

\begin{enumerate}\setcounter{enumi}{3}

\item (5 pts) Which variable do you choose as a state variable? Which variable do you choose as a control variable? Write down the Bellman equation.

\item (5 pts)  Derive the first-order condition associated with the Bellman equation.

\item (5 pts)  Derive the Benveniste-Scheinkman equation.

\item (5 pts)  Derive the Euler equation. Compare it with the Euler equation obtained with the Lagrangian method and discuss.

\item (5 pts) Suppose that the policy function takes the form $h(k)=A\cdot (1+r)\cdot k$ where $A\in (0,1) $. Derive $A$.

\item (10 pts) Suppose that the value function takes the form $V(k)=B+D\cdot \ln(k),$ where $B$ and $D$ are constants. Using the expression for the policy function that you derived in the previous question, derive $B$ and $D$.

\end{enumerate}
\end{document}