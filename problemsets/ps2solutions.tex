\documentclass[letterpaper,12pt,leqno]{article}
\usepackage{paper,math,notes}
\available{https://pascalmichaillat.org/c3/}
\hypersetup{pdftitle={Problem Set on Optimal Control: Solutions}}

\begin{document}

\title{Problem Set on Optimal Control: Solutions}
\author{Pascal Michaillat}
\date{}

\begin{titlepage}
\maketitle
\end{titlepage}

\section*{Solution to Problem 1}

\begin{enumerate}
\item The present-value Hamiltonian is
\begin{align*}
\Hc(t)=e^{-\rho\cdot  t}\cdot \ln{c(t)} +\l(t) \bs{f\bp{k(t)} -c(t)-\d\cdot  k(t)}
\end{align*}
where $\l(t)$ is the co-state variable associated with the state variable $k(t)$.
\item The optimality conditions for the present-value Hamiltonian are
\begin{align*}
\pd{\Hc(t)}{c(t)} &=0\\
\pd{\Hc(t)}{k(t)} &=-\dot{\l}(t)\\
\lim_{t\to+\infty}\l(t)\cdot k(t)&=0.
\end{align*}
The last condition is the transversality condition. The first two conditions imply that 
\begin{align}
e^{-\rho \cdot t}\cdot \frac{1}{c(t)} =\l(t)\label{eq:HAM1} \\
\l(t)\cdot  \bs{f'\bp{k(t)} -\d} =-\dot{\l}(t).\label{eq:HAM2}
\end{align}
We can eliminate $\l(t)$ by taking log and differentiating \eqref{eq:HAM1} with respect to time $t$. This procedure yields
\begin{align*}
\frac{\dot{\l}(t)}{\l(t)}=-\rho -\frac{\dot{c}(t)}{c(t)}
\end{align*}
We can then substitute $\dot{\l}(t)/\l(t)$ into \eqref{eq:HAM2}, which gives the following Euler equation
\begin{align*}
\frac{\dot{c}(t)}{c(t)} &=\a\cdot A\cdot k(t)^{\a -1}-\bp{\d +\rho} .
\end{align*}
\item The steady state is given by
\begin{align*}
k^*&=\bp{\frac{\a\cdot A}{\d +\rho}} ^{1/\bp{1-\a}}\\
c^*&=A^{1/\bp{1-\a}}\bp{\frac{\a}{\d +\rho}}^{\a /\bp{1-\a}}\cdot \bp{\frac{\d \cdot \bp{1-\a} +\rho}{\d +\rho}}.
\end{align*}
\end{enumerate}


\section*{Solution to Problem 2}

\begin{enumerate}
\item The current-value Hamiltonian is
\begin{align*}
\Hc^{*}(t)=f\bp{k(t)}-i(t)-\frac{\chi}{2}\cdot \bp{\frac{i(t)^{2}}{k(t)}}+q(t)\cdot i(t), 
\end{align*}
where $q(t)$ is the co-state variable associated with the state variable $k(t)$.
\item There are two optimality conditions for the current-value Hamiltonian. (We omit the transversality condition.) The first optimality condition is
\begin{align*}
0&=\pd{\Hc^{*}(t)}{i(t)} \\
0&=-1-\chi \cdot \bs{\frac{i(t)}{k(t)}}+q(t) \\
i(t)&=\bs{\frac{q(t)-1}{\chi}}\cdot k(t),
\end{align*}
which implies, using the law of motion of capital, that
\begin{align*}
\dot{k}(t)&=\bs{\frac{q(t)-1}{\chi}}\cdot k(t).
\end{align*}
The second optimality condition is
\begin{align*}
\pd{\Hc(t)}{k(t)} &=r\cdot q(t)-\dot{q}(t)\\
f'(k(t)) +\frac{\chi}{2}\cdot \bs{\frac{i(t)}{k(t)}}^{2}&=r\cdot q(t)-\dot{q}(t).
\end{align*}
The first optimality condition implies that $i(t)/k(t)=\dot{k}(t)/k(t)=(q(t)-1)/\chi$. So this optimality condition becomes
\begin{align*}
\dot{q}(t)&=r\cdot q(t)-f'(k(t))-\frac{1}{2\cdot \chi}\cdot \bp{q(t)-1}^{2}.
\end{align*}
\item In steady state, $\dot{q}(t)=0$ and $\dot{k}(t)=0$, so $i^{*}=0$. Notice that we can say that $\dot{q}(t)=0$ only because $q(t)$ is the co-state variable used with a current-value Hamiltonian. The co-state variables used in a present-value Hamiltonian are not constant in steady state (which is a reason why we prefer to work with a current-value Hamiltonian). Since $\dot{k}(t)=0$, the first optimality condition implies
\begin{align*}
q^{*}=1.
\end{align*}
Since $q^{*}=1$ and $\dot{q}(t)=0$, the second optimality condition implies
\begin{align*} 
f'(k^{*})=r.
\end{align*}
\end{enumerate}

\end{document}